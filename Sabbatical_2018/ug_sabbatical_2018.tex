
%%%%%%%%%%%%%%%%%%%%%%%%%%%%%%%%%%%%%%%%%%%%%%%%%%%%%%%%%%%%%%%%%%%%%%%%%
%                                                                       %
%     Podanie odnosnie platnego urlopu naukowego na UG                  %
%                                                                       %
%                                                                       %
%                                                                       %
%                                                                       %
%%%%%%%%%%%%%%%%%%%%%%%%%%%%%%%%%%%%%%%%%%%%%%%%%%%%%%%%%%%%%%%%%%%%%%%%%

\documentclass[12pt]{article}
\usepackage[polish]{babel}
\usepackage[utf8]{inputenc}
\usepackage{t1enc}

\pagestyle{empty}

\begin{document}
%\magnification\magstep1
\large

\bigskip
\bigskip

{\parindent0cm \noindent
Dr hab. Andrzej Nowik, prof UG  \hfill
Gdańsk, dnia 06.06.2018 \par
Uniwersytet Gdański\\
Instytut Matematyki \\
Wydział Matematyki, Fizyki i Informatyki \\
Wita Stwosza 57 \\
80-952 Gdańsk \\
}

\bigskip
\bigskip
\bigskip


\par
{ \parindent6.2cm
 \par
  Jego Magnificencja
 \par
  Prorektor Uniwersytetu Gdańskiego
 \par 
  ds. Nauki i Współpracy z Zagranicą
 \par
  Prof. dr hab. 
 \par 
  Piotr Stepnowski
}

{
\bigskip
\bigskip
\bigskip
\bigskip
\bigskip
Szanowny Panie Rektorze,

\bigskip
\bigskip

  Uprzejmie proszę o udzielenie mi 
płatnego urlopu naukowego (Art 134 ust. 1 i 3
ustawy Prawo o szkolnictwie wyższym, Dz.U. 2012/572)
w okresie od 1 października 2018 roku do 30 czerwca 2019 roku.
  W tym czasie mam zamiar prowadzić poza swoją macierzystą uczelnią 
badania które dotyczyć będą szeroko pojętych zastosowań deskryptywnej teorii mnogości
w topologii, teorii funkcji rzeczywistych oraz w analizie matematycznej. 
  Podczas wspomnianego urlopu zamierzam współpracować
z grupą badawczą z podstaw matematyki przy 
Uniwersytecie Warszawskim a także
z ośrodkiem badawczym na polu teorii funkcji rzeczywistych
przy Uniwersytecie Kazimierza Wielkiego w Bydgoszczy.
  Planuję regularnie odbywać kilkudniowe pobyty
w Insytucie Matematyki Uniwersytetu Warszawskiego
który jest jednym z wiodących ośrodków zastosowań teorii
mnogości w topologii i teorii miary. Wspomniane 
pobyty mają mieć miejsce w pierwszym semestrze trwania mego urlopu 
(październik 2018 - luty 2019). W ich trakcie
będę między innymi uczestniczyć w seminarium 
zakładowym ,,Teoria Mnogości'' (Uniwersytet Warszawski, środa, godzina 16:15).
  W drugim semestrze (marzec 2019 - czerwiec 2019) 
planuję serię kilkudniowych pobytów
w Instytucie Matematyki Uniwersytetu Kazimierza Wielkiego w Bydgoszczy
i w ich trakcie uczestniczył będę w seminarium
zakładowym Zakładu Topologii (odbywającym się we wtorki).
}
\bigskip
\bigskip
{\parindent8cm \par
Łączę wyrazy szacunku 
}


\end{document}


